\documentclass[UTF8,a4paper,10pt]{ctexart}
\usepackage[left=2.50cm, right=2.50cm, top=2cm, bottom=2.50cm]{geometry} %页边距
\CTEXsetup[format={\Large\bfseries}]{section} %设置章标题居左
 \usepackage{amsmath}
 
 \usepackage{mathtools}
 
 
%%%%%%%%%%%%%%%%%%%%%%%
% -- text font --
% compile using Xelatex
%%%%%%%%%%%%%%%%%%%%%%%
% -- 中文字体 --
%\setmainfont{Microsoft YaHei}  % 微软雅黑
%\setmainfont{YouYuan}  % 幼圆    
%\setmainfont{NSimSun}  % 新宋体
%\setmainfont{KaiTi}    % 楷体
%\setmainfont{SimSun}   % 宋体
%\setmainfont{SimHei}   % 黑体
% -- 英文字体 --
%\usepackage{times}
%\usepackage{mathpazo}
%\usepackage{fourier}
%\usepackage{charter}
\usepackage{helvet}
 
 
\usepackage{amsmath, amsfonts, amssymb} % math equations, symbols
\usepackage[english]{babel}
\usepackage{color}      % color content
\usepackage{graphicx}   % import figures
\usepackage{url}        % hyperlinks
\usepackage{bm}         % bold type for equations
\usepackage{multirow}
\usepackage{booktabs}
\usepackage{epstopdf}
\usepackage{epsfig}
\usepackage{algorithm}
\usepackage{algorithmic}
\renewcommand{\algorithmicrequire}{ \textbf{Input:}}     % use Input in the format of Algorithm  
\renewcommand{\algorithmicensure}{ \textbf{Initialize:}} % use Initialize in the format of Algorithm  
\renewcommand{\algorithmicreturn}{ \textbf{Output:}}     % use Output in the format of Algorithm  
 
 

\usepackage{fancyhdr} %设置页眉、页脚
\pagestyle{fancy}
\lhead{}
\chead{数学分析I \  \ 第二次测验试卷 \ \ 共4页 \ \ 第 \thepage 页}
\rhead{}
%\rhead{\includegraphics[width=1.2cm]{fig/ZJU_BLUE.eps}}
\lfoot{}
\cfoot{}
\rfoot{}
 
 

 
%%%%%%%%%%%%%%%%%%%%%%%
%  设置水印
%%%%%%%%%%%%%%%%%%%%%%%
%\usepackage{draftwatermark}         % 所有页加水印
%\usepackage[firstpage]{draftwatermark} % 只有第一页加水印
% \SetWatermarkText{Water-Mark}           % 设置水印内容
% \SetWatermarkText{\includegraphics{fig/ZJDX-WaterMark.eps}}         % 设置水印logo
% \SetWatermarkLightness{0.9}             % 设置水印透明度 0-1
% \SetWatermarkScale{1}                   % 设置水印大小 0-1    
 
\usepackage{hyperref} %bookmarks
%\hypersetup{colorlinks, bookmarks, unicode} %unicode
 
 
 
%\title{\textbf{数学分析 \  第一次测验试卷 \ 共4页}}
%\author{ Simmel \thanks{学号:xx2017xxxx} }
%\date{}
 
\begin{document}
   % \maketitle
   
   
  \vspace{-2cm}
\begin{center}
{\Large\textbf{数学分析I \  第二次测验试卷 \ 共4页}}


\vspace{0.9cm}

 ( 考试形式 \quad\quad  闭卷 \quad\quad 2021年12月19日)\\
 \bigskip
           院\quad 系\underline{\makebox[30mm][c]{    }}  年\quad 级\underline{\makebox[30mm][c]{    }}  专\quad 业\underline{\makebox[30mm][c]{}}\\
            学\quad 号\underline{\makebox[30mm][c]{    }}  姓\quad 名\underline{\makebox[30mm][c]{    }}  成\quad 绩\underline{\makebox[30mm][c]{}}\\

\end{center}



\noindent 一、(本题20分)概念题.

\begin{itemize}












\item[(1)]
设$f$为定义在区间$I$上的函数. 按$\varepsilon-\delta$的语言叙述$f$在$I$上不一致连续的定义,并由此定义证明函数$f(x)=x^3$在$[1, +\infty)$上不一致连续.











\vspace{5cm}
\item[(2)]
设函数$y=f(x)$在点$x_0$的某邻域内有定义. 叙述$f$在$x_0$处可导的定义,并按定义求$f(x)=x^3$在点$1$处点导数.










\vspace{5cm}
\end{itemize}



\noindent 二、(本题48分)计算题.\\





\noindent
\begin{minipage}{.5\textwidth}
\begin{itemize}
\item[(1)]
求极限$\lim\limits_{x \to 0} \frac{\sin{(2x)}-2\sin {x}}{4x^3}$.
\vspace{5cm}
\end{itemize}  
\end{minipage}
\hfill
\begin{minipage}{.4\textwidth}
\begin{itemize}
\item[(2)]
求$\lim\limits_{x \to 0+} [x]+2[-x]. $
\vspace{5cm}
\end{itemize}
\end{minipage}





\newpage
\noindent
\begin{minipage}{.45\textwidth}
\begin{itemize}
\item[(3)]
求导数$((x^3+1)2^x\log_3(x)))' \ (x > 0)$.
\vspace{5cm}
\item[(5)] 
求高阶导数$(\sin x)^{(2022)}$. 
 \vspace{5cm}
\item[(7)] 
设$y=x^2\cos(2x)$, 求微分$dy$.
\vspace{6cm}
\end{itemize}  
\end{minipage}
\hfill
\begin{minipage}{.55\textwidth}
\begin{itemize}
\item[(4)]
求导数$(x^{2^{x}})' \ (x > 0)$. 
\vspace{5cm}
\item[(6)]
设$f(x)$在$\mathbb{R}$上二阶可导,$y=f(f(x))$, 求$\frac{d^2y}{dx^2}$.
\vspace{5cm}
\item[(8)]
求参数方程$\begin{dcases}
x=2\cos^3t, \\
y=\sin^3t
\end{dcases}$ 在$t=\frac{\pi}{4}$处的法线方程.
\vspace{5.5cm}

\end{itemize}
\end{minipage}



\newpage

\noindent 三、(本题10分)
设函数$f$在$\mathbb{R}$上连续. 若极限$\lim\limits_{x \to \infty} [f(x)-x]$存在,证明$f$在$\mathbb{R}$上一致连续.




\vspace{12cm}


\noindent 四、(本题10分)
利用定义证明对数函数求导公式$(\log_ax)'=\frac{1}{x\ln a} (a>0, a\neq 1, x>0)$.




\vspace{8cm}

\newpage



\noindent 五、(本题6分)
设函数$f$在区间$[0, 1]$上连续且$f(0)=f(1)$. 证明对任意正整数$n \in \mathbb{N}$存在$x_n \in [0, 1]$使得$f(x_n)=f(x_n+\frac{1}{n})$.


\vspace{12cm}


\noindent 六、(本题6分)
定义函数$f(x)=\begin{dcases}
x^2,  \ \ \ x \in \mathbb{Q};\\
-x^2,  \ x \in \mathbb{R} \setminus \mathbb{Q}\end{dcases}
$. 证明$f(x)$ 可导当且仅当$x=0$.


\end{document}  
 